

Fuzz testing is a popular technique for automatic software vulnerability detection 
 \cite{Miller:Fuzz, 5010257, sutton2007fuzzing}.
 However, it suffers from low efficiency when applied to real-world software 
 \cite{neystadt2008automated, godefroid2008automating, ganesh2009taint, cadar2011symbolic, rawat2017vuzzer, stephens2016driller},
  which often has complex input formats, e.g., Portable Document Format (PDF).
  Most of the test cases generated by fuzz testing will be discarded on the shallow surface of such software.
  In order to improve the performance of traditional fuzz testing, 
  coverage-based fuzz testing collects all of the test cases that contribute to the coverage into a seed file queue.
  The fuzzer then generates new test input from the seed queue using genetic methods
  \cite{rawat2017vuzzer, online:afl, stephens2016driller}.
  Although coverage-based fuzz testing is able to discover more paths than traditional fuzz testing, 
  it is nevertheless incapable of triggering bugs that are deeply nested in complex code areas (due to the usage of random mutation).

Recently, dynamic symbolic execution has been employed to improve the efficiency of fuzz testing as a form of hybrid testing \cite{godefroid2012sage, yeh2015craxfuzz, majumdar2007hybrid, pak2012hybrid}.
 In this approach, corner cases that are difficult for fuzzers to cover are generated from dynamic symbolic execution by solving the corresponding path conditions.
 Meanwhile, dynamic symbolic execution can also benefit from the seed files in the fuzzer's seed queue to quickly reach more wider code areas. 
 Driller, which is built on top of the Angr symbolic execution engine \cite{Shoshitaishvili_firmalice-automatic} and AFL fuzzing engine \cite{online:afl}, 
 has attempted to leverage symbolic execution to solve the branches guarded 
 by complex path conditions to avoid saturation of fuzzer \cite{stephens2016driller}. 
 Driller's performance in DARPA's Cyber Grand Challenge (CGC) \cite{online:CGC} demonstrates the potential of these hybrid testing approaches.


In hybrid testing, such as Driller, the performance gain from dynamic symbolic execution is still limited 
 by particular program structures(e.g., symbolic pointers and loops) \cite{schwartz2010all, Boonstoppel:RAP, cadar2011symbolic, baldoni2016survey}. 
 Such structures will quickly generate many states that may not trigger new behaviors
 but result in \textit{state explosion}.
 Moreover, by leveraging dynamic symbolic execution, 
 the seed queue of the fuzzer will quickly reach a large number for modern software. 
 %So when given the testing time budget, the seed queue should be rearranged to make sure 
 %that test case with greater probability of triggering new paths will be scheduled with high priority.

 In this paper, we propose two advanced techniques to improve the efficiency of dynamic symbolic execution assisted hybrid testing.
 
 On one hand, based on the lazy forking technique employed in S2E\cite{chipounov2011s2e}, we concretize symbolic pointers to avoid generating too many states but solve the pending states that are forked from these pointers on demand to cover more branches. An optimization based on AFL's \cite{online:afl} loop bucket mechanism is also introduced to avoid getting stuck in symbolic loops.
 On the other hand, to address the large size of seed queue, 
 we propose a distance based seed selection method for fuzz testing to improve the coverage when testing time is limited. 
 Each seed in the queue is equipped with an weight value, 
 obtained from the execution runtime information,
 which includes both path coverage and memory coverage.
 Our method prioritizes the seed queue according to this weight value and 
 then selects the seed file with the greatest weight value for next mutation cycle.


 Our main contributions consist of two main components, namely \emph{Symbolic Path Finder (SPF)} and \emph{Seacher}. 
 The \emph{SPF} component is leveraged to help the fuzzer dive into deeper code areas 
 that are guarded by complex path constraints. 
 Techniques to handle the \textit{state explosion} problem 
 raised by symbolic pointers and loops are implemented inside of \emph{SPF}. 
 The \emph{Searcher} is designed to select the most promising seed file 
 from the seed queue based on the distance measurement. 
 By doing this, the fuzzer will reach previously untouched code areas as soon as possible in a given time budget. 

 Last but not least,  we have implemented the proposed techniques in a prototype tool,
 and performed comprehensive experimental evaluations on three different benchmarks. The benchmarks consist of: a demo program which contains 9 different types of bugs; the LAVA benchmark suite \cite{dolan2016lava}; and a set of real world UNIX programs. The results show that our prototype can trigger more bugs than other state-of-the-art vulnerability detection tools. We also evaluated the path discovery ability of our prototype on a benchmark which contains several real-world UNIX programs, and the result shows that our approach can discover 43.49\% more unique paths in average than vanilla fuzz testing.

In summary, this paper makes the following contributions:
\begin{itemize}
\item We introduce a technique to avoid forking more states by postponing the concretization of symbolic pointer to the moment when branch condition depends on such pointer.  

\item We also present an optimization namely \emph{symbolic loop bucket} to ease the \textit{state explosion} problem by limiting the looping times into a serial of fixed buckets.

\item A \emph{distance based seed selection} method is proposed to select the most promising seed in the queue according to the runtime information to improve path coverage. 
\end{itemize}


The rest of this paper is organized as follows. 
 Section~\ref{sec:preliminaries} describes the basic conception of dynamic symbolic execution and hybrid testing. 
 Section~\ref{sec:ease PE} presents the details of how we deal with \textit{state explosion} raised from symbolic pointers and loops. The distance based seed selection method is discussed in Section~\ref{sec:seed selection}. Section~\ref{sec:evaluate} describes the implementation of our prototype and the evaluation results. Section~\ref{sec:discussion} discusses the limitations of our work and possible counter measures. Section~\ref{sec:related} reviews the related work, and Section~\ref{sec:conclusion} concludes this paper.
