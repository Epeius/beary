Our prototype is built on top of the state-of-the-art genetic fuzzing framework AFL, which is a popular off-the-shelf fuzzer, and symbolic execution platform S2E. S2E is a dynamic binary analysis platform which utilizes selective symbolic execution to analyze whole software stacks at runtime. So far, S2E is available for many instructions set architecture, such as X86, ARM and so on. S2E reuses parts of the QEMU virtual machine, the KLEE symbolic execution engine and the LLVM toolchain.

The fuzzing logic of AFL was not changed obviously, and we only added some extra code to implement the seed searcher. Firstly, when AFL finds a new seed, the seed searcher calculates the weight according to the distance measures. Secondly, the seed searcher selects the most promising seed file for next mutation. On the S2E side, we implemented three main components, namely \textit{BitmapSearcher}, \textit{SLBhandler} and \textit{LSPhandler}. The \textit{BitmapSearcher} is used to share internal information with AFL to generate test cases for the uncovered branches. The \textit{SLBhandler} focuses on symbolic loops and leverages SLB method to only generate test cases for specific loop times according to the loop buckets. The \textit{LSPhandler} implements the LSP algorithm to generate test cases for branches where state forking failure occur to improve the coverage.
