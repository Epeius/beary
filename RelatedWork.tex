% Related Work.tex
We have presented the major advantages of our method in the previous sections and compared our system with some state-of-the-art vulnerability discovery tools. In this section, we present the techniques that related to our method.

\noindent\textit{\textbf{Similarity Distance in Regression Testing:}}
Similarity based algorithms have been leveraged to regression test case prioritization \cite{wang2015similarity, zhang2012simfuzz, jones2003test}. Test case prioritization issue is a hot research topic in regression testing research, which tries to optimum mutation schedule based on a specific prioritization criterion. Rothermel et, al. proposed fine-grained prioritization strategy based on the instruction coverage and branch coverage \cite{rothermel2001prioritizing}. Then Elbaum et, al. concentrated on function level coverage and they proved that this kind of coarse-grained instrumentation which can reduce the execution overhead but will lose some prioritization performance \cite{elbaum2001incorporating}. Krishna et al. utilized Levenshtein distance as the criterion of prioritization \cite{krishnamoorthi2009factor}. Rather than using an ordered branch sequence to present the path in \cite{wang2015similarity}, we represented the execution path by using the bitmap in AFL, which is more practical and efficiency.

\noindent\textit{\textbf{Taint Analysis based Fuzz Testing:}}
Taint analysis based fuzz testing uses dynamic taint analysis (DTA) to locate regions of seed input that affect the execution path. BuzzFuzz uses DTA to automatically locate regions of original seed input files that influence values used at key program attack points, and then automatically generates new fuzzed test input files by fuzzing these identified regions of the original seed input files \cite{ganesh2009taint}. TaintScope is a directed fuzzing tool working at X86 binary level. Based on fine-grained DTA, TaintScope identifies which bytes in a well-formed input are used in security-sensitive operations (e.g., invoking system/library calls) and then focuses on modifying such bytes. And TaintScope is also capable of bypassing checksums via control flow alteration \cite{wang2010taintscope}. Dowser is a guided fuzzer that combines static analysis, DTA, and symbolic execution to find buffer overflow vulnerabilities deep in a program’s logic, and it ranks pointer dereference instructions according to their complexity, and then uses symbolic execution to zoom in on the most interesting operation \cite{haller2013dowsing}.

\noindent\textit{\textbf{Hybrid Testing Method:}}
As mentioned before, our method is not the first tool to combine fuzz testing and symbolic execution. Hybrid Fuzz Testing uses symbolic execution to discover frontier nodes that represent unique paths in the program \cite{pak2012hybrid}. After collecting as many frontier nodes as possible under a user-specifiable resource constraint, it transits to fuzz the program with random inputs. This tool focuses on binaries but only performs the one-time transition between symbolic execution and fuzz testing. Hybrid Concolic Testing implements multiple transitions between symbolic execution and fuzz testing \cite{majumdar2007hybrid}. But because it is built on top of CUTE, a source code oriented testing tool, so hybrid concolic testing still cannot be deployed on binary testing directly \cite{sen2005cute}. Driller is an up-to-date hybrid testing tool that leverages fuzz testing and concolic execution in a complementary manner to find deeper bugs \cite{stephens2016driller}. It is more practice when compared with previous hybrid tools. Some other tools try to make full use of symbolic execution to maximize the code coverage, they collect symbolic constraints placed on each input and then negating these constraints to generate a new test case that will take another uncovered path, such as SAGE \cite{godefroid2012sage}, Dowser \cite{haller2013dowsing}, FuzzWin \cite{online:fuzzwin} etc. However, as these tools execute each input in the symbolic mode which determines that they have to face the ``state explosion'' problem. 